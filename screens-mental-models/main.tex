\documentclass[a4paper]{article}

\usepackage[english]{babel}
\usepackage[utf8]{inputenc}
\usepackage{amsmath}
\usepackage{graphicx}
\usepackage[colorinlistoftodos]{todonotes}

\title{Screens \& Their Mental Models}

\author{Joseph Barbosa}

\date{\today}

\begin{document}
\maketitle

\begin{abstract}

     Display technologies have been around for years to assist users of many different technologies. They serve as the aspect of technology which interacts with the user both visually and manually. It used to be the case that screens existed with the sole purpose of displaying infformation to users and the users interaction with screens was solely visual; this has changed. Over the past three decades touch screen technology has advanced and become more common, from computers to PDA's to phones to everyday machines, adding the manual interaction aspect. I seek to evaluate screens and the mental models associated with them and their interaction design principles, in particular those associated with touch screens. Touch screens are an ever-advancing technology in today's world and they continue to evolve to meet users wants and needs.

\end{abstract}

\section{Introduction}

     Screens are everywhere, and used more often than people would like to believe. Screens, as defined by Google, are ''a flat panel or area on an electronic device such as a television, computer, or smartphone, on which images and data are displayed''. Screens have become a modern commodity pivotal to everyday life, and as technology advances, as must screens. Screens have always been a means of interaction between users and the device whose screen is being interacted with. The interaction between screens and user began as a visual interaction, with data being displayed to the user, but this has changed. Screens have incorporated a manual part, bringing human  physical interaction into the picture and making the mental model for interaction design one that focuses not only on visual aspects but manual aspects too. 

\section{Background}

     Over the past ten years, many advancements have been made in the field of touch screens, and along with those advancements came new studies. As the technology advances, inventors, programmers, and manufacturers all look to make their products more user-friendly, so they conduct studies. Studies have been done for almost every instance of touch screen technology used widely and the results all seem to lead to the same results, which have their own ups and downs.
     
     Touch screens are used for everything from phones to ATM machines to cars to computers. The screens themselves and their purposes may be different for each case yet they all follow the same Interaction design principles and similar mental models. Some studies that have been conducted which look at the touch screen interaction mental model and the results are surprising. These studies look at such aspects of touch screen interaction as how the screens and user physically interact (touch), how gestures and action cater to the mindset of the user, as well as some misconceptions about touch. 
     
     In one study, conducted by the Department of Computer Science at the University of Maryland and the Department of CISE at the University of Florida, gestures were studied amongst both children and adult users. The purpose of the study was to determine the best user-defined gesture approach to touch screens through testing of individuals with different mental capabilities. Since these technologies do intend to cater to the user and the users of touch screens has expanded to children, the mental model needed to change. The learnability of touch screens and their gestures was too much for some users, children and even some adults, as they had trouble learning new gestures on certain touch screens. This was a result of previous user interactions with touch screens and how the gestures for each vary, if the users had ever used any prior. the results indicated that both children and adults alike during this study showed a common mindset when it came to which gestures came naturally and which did not. 
     
     Just as in the study of gestures, the rant on the future of interaction design article chose to study how users interact with touch screens and what their mindsets deemed natural and what is easiest for users to interact with. The article goes into detail about how designing interfaces for the future now revolves around touch screen technology and how past user interactions and user mental models differ from what is to come from the shift to touch screens. The past has seen such interaction aspects as physical elements (keyboards, mice, buttons, controls) and for this many users worked with their hands and tangible aspects of the established interaction design principles. Such aspects as Fitts's Law no longer apply since we do not have to deal with a mouse movin across the screen, it is simply how far elements are from the home button or power button or, depending on how the user interacts with the screen, the side or wherever they might be grasping it. This also alters such aspects as Human Interface objects, due to the loss of the physical interaction of a mouse, the user may be more prone to errors when clicking on the screen since there is no visual aid of a pointer or any stimulus on the hand except that of making contact with the screen, which could be acciedental. Also, such aspects as hover and right-clicking are lost in translation to some touch screen technologies and this requires new gestures and further alters the mental model for such technologies. 
     
     Lastly, with touch screen technologies there is a new factor which must be considered, and that is the user. Not every user is the same and some have different touch strengths, finger size, and even visiblity. Some of the features on touch screens depend on such things as angles of vision, centroids of touch, and finger strength or even size. These are all factors which are affecting the mental models of touch screen technology and altering how interaction corresponds to some interaction design principles. 

\section{Discussion}

     While reading these studies and considering what they have to say juxtaposed to what we talk about in class i came up with my own observations about touch screen technologies and how their mental models are changing according to interaction design principles. Touch screen technology can still follow the principles previously established, yet it must be done so in a whole new manner based on each unique screen. Each screen designed for interaction may have different aspects and needs for the user thus it may need to follow principles for something like human interface objects, while others might. But there is a commonality to all of these technologies which is in how the user uses their hands and eyes to interact with the screen and this differs from previous manners of interaction thus learnibility may be hindered and users may not feel comfortable with certain gestures and being limited on what to do with their hands. This is something that requires a new adaptation of the principles and new mental models to be developed for new technologies or advancements to be more user friendly.
     
\section{Conclusion}

     As screens have changed over the years, so have their implementations and user mental models. The change from solely visual interaction to visual and manual is a big one and this plays a key role in the interaction design principles for screen technologies and how users will cope with their use. There will exist varying mental models and vaiations of existing models and design principles for each new technology until enough studies have been done and researched such that a new set of principles or models can be adapted for such technologies.
     
\end{document}